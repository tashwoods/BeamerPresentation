%%%%%%%%%%%%%%%%%%%%%%%%%%%%%%%%%%%%%%%%%
% Beamer Presentation
% LaTeX Template
% Version 1.0 (10/11/12)
%
% This template has been downloaded from:
% http://www.LaTeXTemplates.com
%
% License:
% CC BY-NC-SA 3.0 (http://creativecommons.org/licenses/by-nc-sa/3.0/)
%
%%%%%%%%%%%%%%%%%%%%%%%%%%%%%%%%%%%%%%%%%

%----------------------------------------------------------------------------------------
%	PACKAGES AND THEMES
%----------------------------------------------------------------------------------------

\documentclass{beamer}
\setbeamertemplate{caption}{\raggedright\insertcaption\par}

\usepackage{enumitem}
\mode<presentation> {
\usetheme{boxes}
\setbeamertemplate{footline}[page number] % To replace the footer line in all slides with a simple slide count uncomment this line
}

\usepackage{graphicx} % Allows including images
\usepackage{booktabs} % Allows the use of \toprule, \midrule and \bottomrule in tables
\usepackage{caption}
\captionsetup[figure]{labelformat=empty}% redefines the caption setup of the figures environment in the beamer class.

%----------------------------------------------------------------------------------------
%	TITLE PAGE
%----------------------------------------------------------------------------------------

\title[Third Stage SEIR]{Third Stage SEIR} % The short title appears at the bottom of every slide, the full title is only on the title page

%\author[Natasha Woods, Brett Boval, Eric Carlson]{Natasha Woods, Brett Boval, Eric Carlson} % Your name
%\institute[UCSC] % Your institution as it will appear on the bottom of every slide, may be shorthand to save space
%{
%University of California Santa Cruz\\ % Your institution for the title page
%\medskip
%\textit{nlwoods@ucsc.edu} % Your email address
%}
\date{\today} % Date, can be changed to a custom date
%\newcommand{\old}{api-370-prod/pyseir/state_summaries/reports/}
\newcommand{\old}{current_two_stage_output/pyseir/state_summaries/reports/}
\newcommand{\new}{new_shortest_t_delta/pyseir/state_summaries/reports/}
\newcommand{\az}{Arizona__04}
\newcommand{\ak}{Arkansas__05}
\newcommand{\ca}{California__06}
\newcommand{\co}{Colorado__08}
\newcommand{\ct}{Connecticut__09}
\newcommand{\de}{Delaware__10}
%\newcommand{\dcolumb}{District of Columbia__11)
\newcommand{\fl}{Florida__12}
\newcommand{\ga}{Georgia__13}
\newcommand{\hi}{Hawaii__15}
\newcommand{\id}{Idaho__16}
\newcommand{\il}{Illinois__17}
\newcommand{\indiana}{Indiana__18}
\newcommand{\ia}{Iowa__19}
\newcommand{\ks}{Kansas__20}
\newcommand{\ky}{Kentucky__21}
\newcommand{\la}{Louisiana__22}
\newcommand{\me}{Maine__23}
\newcommand{\md}{Maryland__24}
\newcommand{\ma}{Massachusetts__25}
\newcommand{\mi}{Michigan__26}
\newcommand{\mn}{Minnesota__27}
\newcommand{\ms}{Mississippi__28}
\newcommand{\mo}{Missouri__29}
\newcommand{\mt}{Montana__30}
\newcommand{\neb}{Nebraska__31}
\newcommand{\nv}{Nevada__32}
\newcommand{\nh}{New Hampshire__33}
\newcommand{\nj}{New Jersey__34}
\newcommand{\nm}{New Mexico__35}
\newcommand{\ny}{New York__36}
\newcommand{\nc}{North Carolina__37}
\newcommand{\nd}{North Dakota__38}
\newcommand{\oh}{Ohio__39}
\newcommand{\ok}{Oklahoma__40}
\newcommand{\org}{Oregon__41}
\newcommand{\pa}{Pennsylvania__42}
\newcommand{\ri}{Rhode Island__44}
\newcommand{\scal}{South Carolina__45}
\newcommand{\sd}{South Dakota__46}
\newcommand{\tn}{Tennessee__47}
\newcommand{\tx}{Texas__48}
\newcommand{\ut}{Utah__49}
\newcommand{\vt}{Vermont__50}
\newcommand{\va}{Virginia__51}
\newcommand{\wa}{Washington__53}
\newcommand{\wv}{West Virginia__54}
\newcommand{\wi}{Wisconsin__55}
\newcommand{\wy}{Wyoming__56}

\newcommand\makeslide[1]{
\begin{frame}
%\frametitle{ #1 }
    \begin{columns}
     \column{0.5\textwidth}
     %\centering
       \includegraphics[width=6cm, height=3.5cm]{\old mle_fit_results__#1.pdf} \\  
        \includegraphics[width=4.5cm, height=2cm]{\old mle_fit_model__#1.pdf}
       \captionof{figure}{Two Stages}    
        \column{0.5\textwidth}
        %\centering
       \includegraphics[width=6cm, height=3.5cm]{\new mle_fit_results__#1.pdf}  \\ 
        \includegraphics[width=4.5cm, height=2cm]{\new mle_fit_model__#1.pdf}
       \captionof{figure}{Three Stages}    
\end{columns}
\end{frame}
}


\begin{document}

\begin{frame}
\titlepage % Print the title page as the first slide
\end{frame}

%vim processing ls mle_fit_results*pdf > files.txt, open file in vim and
%     :%s/.\{-}__//
%     %s/.\{4}$//
%     %s!^!\\makeslide{!
%     %s/$/\}

%\begin{frame}
%\frametitle{Overview} % Table of contents slide, comment this block out to remove it
%\tableofcontents % Throughout your presentation, if you choose to use \section{} and \subsection{} commands, these will automatically be printed on this slide as an overview of your presentation
\setbeamertemplate{section in toc}{\inserttocsectionnumber.~\inserttocsection}
\begin{frame}
\tableofcontents
\end{frame}


%----------------------------------------------------------------------------------------
%	PRESENTATION SLIDES
%----------------------------------------------------------------------------------------

\section{Why a Third Stage is needed} % Sections can be created in order to organize your presentation into discrete blocks, all sections and subsections are automatically printed in the table of contents as an overview of the talk
%------------------------------------------------
\begin{frame}
\frametitle{Why a Third Stage is Needed}
\begin{itemize}[label={-}]
\item Current Implementation fits two only two stages (unmitigated growth period and linear ramp to $R_{eff}$)
\item As areas across the US reopen, this model will fail to fit the increased R values

\end{itemize}
\end{frame}

\section{Implementation} % Sections can be created in order to organize your presentation into direte blocks, all sections and subsections are automatically printed in the table of contents as an overview of the talk
%------------------------------------------------
\begin{frame}
\frametitle{Implementation}
\begin{itemize}[label={-}]
\item modified modelfitter
\item suppressionpolicies
\item ensemblegenerator
\end{itemize}
\end{frame}


%%%%%%%%%%%%%%%%%%%%%%%%%%%%%%%%%%%%%
\section{State Comparisons}
\subsection{State Comparisons} % A subsection can be created just before a set of slides with a common theme to further break down your presentation into chunks
\begin{frame}
\begin{center}
\frametitle{State Comparisons}
\end{center}
\end{frame}
\makeslide{Alabama__01}
\makeslide{Alaska__02}
\makeslide{Arizona__04}
\makeslide{Arkansas__05}
\makeslide{California__06}
\makeslide{Colorado__08}
\makeslide{Connecticut__09}
\makeslide{Delaware__10}
\makeslide{District of Columbia__11}
\makeslide{Florida__12}
\makeslide{Georgia__13}
\makeslide{Hawaii__15}
\makeslide{Idaho__16}
\makeslide{Illinois__17}
\makeslide{Indiana__18}
\makeslide{Iowa__19}
\makeslide{Kansas__20}
\makeslide{Kentucky__21}
\makeslide{Louisiana__22}
\makeslide{Maine__23}
\makeslide{Maryland__24}
\makeslide{Massachusetts__25}
\makeslide{Michigan__26}
\makeslide{Minnesota__27}
\makeslide{Mississippi__28}
\makeslide{Missouri__29}
\makeslide{Montana__30}
\makeslide{Nebraska__31}
\makeslide{Nevada__32}
\makeslide{New Hampshire__33}
\makeslide{New Jersey__34}
\makeslide{New Mexico__35}
\makeslide{New York__36}
\makeslide{North Carolina__37}
\makeslide{North Dakota__38}
\makeslide{Ohio__39}
\makeslide{Oklahoma__40}
\makeslide{Oregon__41}
\makeslide{Pennsylvania__42}
\makeslide{Rhode Island__44}
\makeslide{South Carolina__45}
\makeslide{South Dakota__46}
\makeslide{Tennessee__47}
\makeslide{Texas__48}
\makeslide{Utah__49}
\makeslide{Vermont__50}
\makeslide{Virginia__51}
\makeslide{Washington__53}
\makeslide{West Virginia__54}
\makeslide{Wisconsin__55}
\makeslide{Wyoming__56}

\begin{frame}
\frametitle{Alabama}
    \begin{columns}[t]
     \column{0.5\textwidth}
       \includegraphics[width=\columnwidth, height=0.32\textheight, keepaspectratio]{\old mle_fit_model__Alabama__01.pdf}
       \includegraphics[width=\columnwidth, height=0.32\textheight, keepaspectratio]{\old mle_fit_results__Alabama__01.pdf}   
       \captionof{figure}{Two Stages}    
        \column{0.5\textwidth}
       \includegraphics[width=\columnwidth, height=0.32\textheight, keepaspectratio]{\new mle_fit_model__Alabama__01.pdf}
       \includegraphics[width=\columnwidth, height=0.32\textheight, keepaspectratio]{\new mle_fit_results__Alabama__01.pdf}   
        \captionof{figure}{Three Stages}       
\end{columns}
\end{frame}


\section{County Comparisons}
\begin{frame}
\begin{center}
\frametitle{County Comparisons}
\end{center}
\end{frame}
\section{Thoughts and Plans} % Sections can be created in order to organize your presentation into direte blocks, all sections and subsections are automatically printed in the table of contents as an overview of the talk
%------------------------------------------------
\begin{frame}
\frametitle{Thoughts and Plans}
\begin{itemize}[label={-}]

\item check counties

\end{itemize}
\end{frame}

\end{document} 
