%%%%%%%%%%%%%%%%%%%%%%%%%%%%%%%%%%%%%%%%%
% Beamer Presentation
% LaTeX Template
% Version 1.0 (10/11/12)
%
% This template has been downloaded from:
% http://www.LaTeXTemplates.com
%
% License:
% CC BY-NC-SA 3.0 (http://creativecommons.org/licenses/by-nc-sa/3.0/)
%
%%%%%%%%%%%%%%%%%%%%%%%%%%%%%%%%%%%%%%%%%

%----------------------------------------------------------------------------------------
%	PACKAGES AND THEMES
%----------------------------------------------------------------------------------------

\documentclass{beamer}
\setbeamertemplate{caption}{\raggedright\insertcaption\par}
%\def\labelitemi{--}
%\def\img#1{%
%\begin{figure}
%\includegraphics[width=\linewidth, height=.2\textheight, keepaspectratio]% 
%{#1} 
%\caption{Caption of \MakeUppercase #1}
%\end{figure}}
\usepackage{enumitem}
\mode<presentation> {

% The Beamer class comes with a number of default slide themes
% which change the colors and layouts of slides. Below this is a list
% of all the themes, uncomment each in turn to see what they look like.

\usetheme{boxes}

% As well as themes, the Beamer class has a number of color themes
% for any slide theme. Uncomment each of these in turn to see how it
% changes the colors of your current slide theme.


%\setbeamertemplate{footline} % To remove the footer line in all slides uncomment this line
\setbeamertemplate{footline}[page number] % To replace the footer line in all slides with a simple slide count uncomment this line

%\setbeamertemplate{navigation symbols}{} % To remove the navigation symbols from the bottom of all slides uncomment this line
}

\usepackage{graphicx} % Allows including images
\usepackage{booktabs} % Allows the use of \toprule, \midrule and \bottomrule in tables
\usepackage{caption}
\captionsetup[figure]{labelformat=empty}% redefines the caption setup of the figures environment in the beamer class.

%----------------------------------------------------------------------------------------
%	TITLE PAGE
%----------------------------------------------------------------------------------------

\title[Third Stage SEIR]{Third Stage SEIR} % The short title appears at the bottom of every slide, the full title is only on the title page

\author[Natasha Woods, Brett Boval, Eric Carlson]{Natasha Woods, Brett Boval, Eric Carlson} % Your name
%\institute[UCSC] % Your institution as it will appear on the bottom of every slide, may be shorthand to save space
%{
%University of California Santa Cruz\\ % Your institution for the title page
%\medskip
%\textit{nlwoods@ucsc.edu} % Your email address
%}
\date{\today} % Date, can be changed to a custom date
\newcommand{\old}{api-370-prod/pyseir/state_summaries/reports/}
\newcommand{\new}{new/pyseir/state_summaries/reports/}

\newcommand{\az}{Arizona__04}
\newcommand{\ak}{Arkansas__05}
\newcommand{\ca}{California__06}
\newcommand{\co}{Colorado__08}
\newcommand{\ct}{Connecticut__09}
\newcommand{\de}{Delaware__10}
\newcommand{\fl}{Florida__12}
\newcommand{\ga}{Georgia__13}
\newcommand{\hi}{Hawaii__15}
\newcommand{\id}{Idaho__16}
\newcommand{\il}{Illinois__17}
\newcommand{\indiana}{Indiana__18}
\newcommand{\ia}{Iowa__19}
\newcommand{\ks}{Kansas__20}
\newcommand{\ky}{Kentucky__21}

\begin{document}

\begin{frame}
\titlepage % Print the title page as the first slide
\end{frame}

%\begin{frame}
%\frametitle{Overview} % Table of contents slide, comment this block out to remove it
%\tableofcontents % Throughout your presentation, if you choose to use \section{} and \subsection{} commands, these will automatically be printed on this slide as an overview of your presentation
\setbeamertemplate{section in toc}{\inserttocsectionnumber.~\inserttocsection}
\begin{frame}
\tableofcontents
\end{frame}


%----------------------------------------------------------------------------------------
%	PRESENTATION SLIDES
%----------------------------------------------------------------------------------------

\section{Why a third Stage is needed} % Sections can be created in order to organize your presentation into discrete blocks, all sections and subsections are automatically printed in the table of contents as an overview of the talk
%------------------------------------------------
\begin{frame}
\frametitle{Why a Third Stage is Needed}
\begin{itemize}[label={-}]
\item Current Implementation fits two only two stages (unmitigated growth period and linear ramp to $R_{eff}$)
\item As areas across the US reopen, this model will fail to fit the increased R values

\end{itemize}
\end{frame}

\section{Implementation} % Sections can be created in order to organize your presentation into discrete blocks, all sections and subsections are automatically printed in the table of contents as an overview of the talk
%------------------------------------------------
\begin{frame}
\frametitle{Implementation}
\begin{itemize}[label={-}]
\item modified modelfitter
\item suppressionpolicies
\item ensemblegenerator

\end{itemize}
\end{frame}

\section{State Comparisons}
%\subsection{Asimov CRs} % A subsection can be created just before a set of slides with a common theme to further break down your presentation into chunks



\begin{frame}
\frametitle{Alabama}
    \begin{columns}[t]

     \column{0.5\textwidth}
       \includegraphics[width=\columnwidth, height=0.32\textheight, keepaspectratio]{\old mle_fit_model__Alabama__01.pdf}
       \includegraphics[width=\columnwidth, height=0.32\textheight, keepaspectratio]{\old mle_fit_results__Alabama__01.pdf}   
       \captionof{figure}{Two Stages}    
        \column{0.5\textwidth}
       \includegraphics[width=\columnwidth, height=0.32\textheight, keepaspectratio]{\new mle_fit_model__Alabama__01.pdf}
       \includegraphics[width=\columnwidth, height=0.32\textheight, keepaspectratio]{\new mle_fit_results__Alabama__01.pdf}   
        \captionof{figure}{Three Stages}    
       
\end{columns}
\end{frame}

\begin{frame}
\frametitle{Alaska}
    \begin{columns}[t]

     \column{0.5\textwidth}
       \includegraphics[width=\columnwidth, height=0.32\textheight, keepaspectratio]{\old mle_fit_model__Alaska__02.pdf}
       \includegraphics[width=\columnwidth, height=0.32\textheight, keepaspectratio]{\old mle_fit_results__Alaska__02.pdf}   
       \captionof{figure}{Two Stages}    
        \column{0.5\textwidth}
       \includegraphics[width=\columnwidth, height=0.32\textheight, keepaspectratio]{\new mle_fit_model__Alaska__02.pdf}
       \includegraphics[width=\columnwidth, height=0.32\textheight, keepaspectratio]{\new mle_fit_results__Alaska__02.pdf}   
        \captionof{figure}{Three Stages}    
       
\end{columns}
\end{frame}


\begin{frame}
\frametitle{Arizona}
    \begin{columns}[t]

     \column{0.5\textwidth}
       \includegraphics[width=\columnwidth, height=0.32\textheight, keepaspectratio]{\old mle_fit_model__\az .pdf}
       \includegraphics[width=\columnwidth, height=0.32\textheight, keepaspectratio]{\old mle_fit_results__\az .pdf}   
       \captionof{figure}{Two Stages}    
        \column{0.5\textwidth}
       \includegraphics[width=\columnwidth, height=0.32\textheight, keepaspectratio]{\new mle_fit_model__\az .pdf}
       \includegraphics[width=\columnwidth, height=0.32\textheight, keepaspectratio]{\new mle_fit_results__\az .pdf}   
        \captionof{figure}{Three Stages}    
       
\end{columns}
\end{frame}

\begin{frame}
\frametitle{Arkansas}
    \begin{columns}[t]

     \column{0.5\textwidth}
       \includegraphics[width=\columnwidth, height=0.32\textheight, keepaspectratio]{\old mle_fit_model__\ak .pdf}
       \includegraphics[width=\columnwidth, height=0.32\textheight, keepaspectratio]{\old mle_fit_results__\ak .pdf}   
       \captionof{figure}{Two Stages}    
        \column{0.5\textwidth}
       \includegraphics[width=\columnwidth, height=0.32\textheight, keepaspectratio]{\new mle_fit_model__\ak .pdf}
       \includegraphics[width=\columnwidth, height=0.32\textheight, keepaspectratio]{\new mle_fit_results__\ak .pdf}   
        \captionof{figure}{Three Stages}    
       
\end{columns}
\end{frame}

\begin{frame}
\frametitle{California}
    \begin{columns}[t]

     \column{0.5\textwidth}
       \includegraphics[width=\columnwidth, height=0.32\textheight, keepaspectratio]{\old mle_fit_model__\ca .pdf}
       \includegraphics[width=\columnwidth, height=0.32\textheight, keepaspectratio]{\old mle_fit_results__\ca .pdf}   
       \captionof{figure}{Two Stages}    
        \column{0.5\textwidth}
       \includegraphics[width=\columnwidth, height=0.32\textheight, keepaspectratio]{\new mle_fit_model__\ca .pdf}
       \includegraphics[width=\columnwidth, height=0.32\textheight, keepaspectratio]{\new mle_fit_results__\ca .pdf}   
        \captionof{figure}{Three Stages}    
       
\end{columns}
\end{frame}


\begin{frame}
\frametitle{Colorado}
    \begin{columns}[t]

     \column{0.5\textwidth}
       \includegraphics[width=\columnwidth, height=0.32\textheight, keepaspectratio]{\old mle_fit_model__\co .pdf}
       \includegraphics[width=\columnwidth, height=0.32\textheight, keepaspectratio]{\old mle_fit_results__\co .pdf}   
       \captionof{figure}{Two Stages}    
        \column{0.5\textwidth}
       \includegraphics[width=\columnwidth, height=0.32\textheight, keepaspectratio]{\new mle_fit_model__\co .pdf}
       \includegraphics[width=\columnwidth, height=0.32\textheight, keepaspectratio]{\new mle_fit_results__\co .pdf}   
        \captionof{figure}{Three Stages}    
       
\end{columns}
\end{frame}

\begin{frame}
\frametitle{Connecticut}
    \begin{columns}[t]

     \column{0.5\textwidth}
       \includegraphics[width=\columnwidth, height=0.32\textheight, keepaspectratio]{\old mle_fit_model__\ct .pdf}
       \includegraphics[width=\columnwidth, height=0.32\textheight, keepaspectratio]{\old mle_fit_results__\ct .pdf}   
       \captionof{figure}{Two Stages}    
        \column{0.5\textwidth}
       \includegraphics[width=\columnwidth, height=0.32\textheight, keepaspectratio]{\new mle_fit_model__\ct .pdf}
       \includegraphics[width=\columnwidth, height=0.32\textheight, keepaspectratio]{\new mle_fit_results__\ct .pdf}   
        \captionof{figure}{Three Stages}    
       
\end{columns}
\end{frame}

\begin{frame}
\frametitle{Delaware}
    \begin{columns}[t]

     \column{0.5\textwidth}
       \includegraphics[width=\columnwidth, height=0.32\textheight, keepaspectratio]{\old mle_fit_model__\de .pdf}
       \includegraphics[width=\columnwidth, height=0.32\textheight, keepaspectratio]{\old mle_fit_results__\de .pdf}   
       \captionof{figure}{Two Stages}    
        \column{0.5\textwidth}
       \includegraphics[width=\columnwidth, height=0.32\textheight, keepaspectratio]{\new mle_fit_model__\de .pdf}
       \includegraphics[width=\columnwidth, height=0.32\textheight, keepaspectratio]{\new mle_fit_results__\de .pdf}   
        \captionof{figure}{Three Stages}    
       
\end{columns}
\end{frame}

\iffalse


\newcommand{\de}{Delaware__10}
\newcommand{\fl}{Florida__12}
\newcommand{\ga}{Georgia__13}
\newcommand{\hi}{Hawaii__15}
\newcommand{\id}{Idaho__16}
\newcommand{\il}{Illinois__17}
\newcommand{\indiana}{Indiana__18}
\newcommand{\ia}{Iowa__19}
\newcommand{\ks}{Kansas__20}
\newcommand{\ky}{Kentucky__21}

\fi







\begin{frame}
\frametitle{Thoughts + Plans}


\begin{itemize}[label={-}]
\item Check counties
\item Test other tbreak boundaries?
\end{itemize}
\end{frame}
\iffalse
\begin{frame}
\frametitle{Alabama}
       \includegraphics[width=\columnwidth, height=0.9\textheight, keepaspectratio]{\new mle_fit_model__Alabama__01.pdf}
\end{frame}


\begin{frame}
\frametitle{Alabama}
    \begin{columns}[t]
    \column{0.5\textwidth}
     \includegraphics[width=\columnwidth, height=0.9\textheight, keepaspectratio]{\old mle_fit_model__Alabama__01.pdf}
      \captionof{figure}{Two Stages}       
      \column{0.5\textwidth}
      \includegraphics[width=\columnwidth, height=0.9\textheight, keepaspectratio]{\new mle_fit_model__Alabama__01.pdf}
       \captionof{figure}{Three Stages}
\end{columns}
\end{frame}
\fi
\end{document} 
